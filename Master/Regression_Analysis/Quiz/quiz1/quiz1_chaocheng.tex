\documentclass[a4paper,12pt]{article}
\usepackage{geometry}
\geometry{left=2.5cm,right=2.5cm,top=2.5cm,bottom=2.5cm}
\renewcommand{\textfraction}{0.15}
\renewcommand{\topfraction}{0.85}
\renewcommand{\bottomfraction}{0.65}
\renewcommand{\floatpagefraction}{0.60}
\usepackage{amsmath}
\usepackage{amsfonts}
\usepackage{mathrsfs}
\usepackage{amsthm}
\usepackage{extarrows}
\usepackage{bm}
\usepackage{graphicx}
\usepackage[section]{placeins}
\usepackage{flafter}
\usepackage{array}
\usepackage{caption}
\usepackage{subcaption}


\title{Quiz 1}
\author{Chao Cheng\quad No.2014210744 }
\date{\today}

\begin{document}

\maketitle

\section*{Analysis}

A study was made on the effect of iron content on the corossion of an alloy.
A linear model $y_i=\beta_0+\beta_1 x_i +\epsilon_i$ is assumed.

\begin{enumerate}
\item Calculate the estimators $b_0$ for $\beta_0$ and $b_1$ for $\beta_1$.
\par
From the fitted model we have $b_0=129.79$ and $b_1=-24.02$. See Table~\ref{tab:summary}.
\par

\item Calculate the standard errors of $b_0$ for $\beta_0$
\par
\begin{table}[htbp]
  \centering
  \begin{tabular}{rrrrr}
  \hline
 & Estimate & Std. Error & t value & Pr($>$$|$t$|$) \\ 
  \hline
$\beta_0$ & 129.7866 & 1.4027 & 92.52 & 0.0000 \\ 
 $\beta_1$ & -24.0199 & 1.2798 & -18.77 & 0.0000 \\ 
   \hline
\end{tabular}
  \caption{Summary table of the fitted model}
  \label{tab:summary}
\end{table}
\par
The standard error of $b_0$ is 1.403 and the standard error of $b_1$ is 1.280. See Table~\ref{tab:summary} for details.
\par



\item Test $H_0:\beta_1=0$ and $H_1:\beta_1\neq 0$ using the $t$-test.
Calculate the $t$ statistic and p-value.
\par
The $t$ statistic is -18.77 and the corresponding p-value is $1.06\times 10^{-9}$. See Table~\ref{tab:summary} for details.
\par



\item Test $H_0:\beta_0=0$ and $H_1:\beta_0\neq 0$ using the $t$-test.
Calculate the $t$ statistic and p-value.
\par
The $t$ statistic is 92.52 and the corresponding p-value is less than $2\times 10^{-16}$. See Table~\ref{tab:summary} for details.
\par




\item Calculate the 99\% confidence interval for
$m(x)=\beta_0+\beta_1 x$ for $x=0$, $x=-5$ and $x=5$.
\par
See Table~\ref{tab:confidence} for the details.
\begin{table}[htbp]
  \centering
  \begin{tabular}{rrrr}
  \hline
$X$ & fitted value & lower bound & upper bound \\ 
  \hline
0 & 129.79 & 125.43 & 134.14 \\ 
 -5 & 249.89 & 226.39 & 273.38 \\ 
  5 & 9.69 & -6.93 & 26.30 \\ 
   \hline
\end{tabular}
  \caption{99\% confidence interval}
  \label{tab:confidence}
\end{table}
\par


\item Present the analysis of variance table.
\par
\begin{table}[htbp]
    \centering
    \begin{tabular}{lrrrrr}
  \hline
 & Df & Sum Sq & Mean Sq & F value & Pr($>$F) \\ 
  \hline
X & 1 & 3293.77 & 3293.77 & 352.27 & 0.0000 \\ 
  Residuals & 11 & 102.85 & 9.35 &  &  \\ 
   \hline
\end{tabular}
    \caption{Anova of the fitted model}
    \label{tab:anova}
\end{table}
\par
See Table~\ref{tab:anova} for the analysis of variance table.
\par

\item Perform the overall $F$ test.
\par
The $F$ statistic is 352.27 with degree of freedom 1 and 11, and the corresponding p-value is $1.05\times 10^{-9}$. See Table~\ref{tab:anova} for details.
\par



\item Plot the scatter plot of $y$ against $x$.
Add the regression line. Add the fitted values against $x$
\par
\begin{figure}[htbp]
  \centering
  \includegraphics[width=0.8\textwidth]{scatterplot}
  \caption{Scatter plot of the data}
  \label{fig:scatterplot}
\end{figure}
\par
See Figure~\ref{fig:scatterplot} for details. The red dots are the fitted values and the white circles are the data points.
\par

\item Plot the residuals against the fitted values.
  \begin{figure}[htbp]
    \centering
    \includegraphics[width=0.8\textwidth]{residual}
    \caption{Residuals against fitted values}
    \label{fig:residuals}
  \end{figure}
\par
See Figure~\ref{fig:residuals} for details. There seems to be a curve pattern in this figure which means the linear model might not be very good. But given the hard truth that there are only 12 obeservations gathered which is a fairly small dataset, we must treat that conclusion very cautiously. And a better thing to do here is to try to collect more data. 
\end{enumerate}

%%%%%%%%%%%%%%%%%%%%%%%%%%%%%%%%%%%%%%%%%%%%%%%%%%%%
\clearpage
\appendix
\section*{Appendix: R codes}
\begin{verbatim}
require(xtable)
Data<-read.table("dat.txt",header=T)
str(Data)
fit.lm<-lm(Y~X,data=Data)

summary(fit.lm)
xtable(summary(fit.lm))

new.x<-data.frame(X=c(0,-5,5))
conf.Y<-predict(fit.lm,newdata = new.x,interval = "confidence",level = 0.99)
xtable(conf.Y)

anova(fit.lm)
xtable(anova(fit.lm))

pdf("scatterplot.pdf")
with(Data,plot(X,Y))
abline(fit.lm)
points(Data$X,fit.lm$fitted.values,pch=16,cex=0.75,col="red")
dev.off()

pdf("residual.pdf")
plot(fit.lm$fitted.values,fit.lm$residuals,xlab="fitted values",ylab="residulas")
abline(h=0,lty=2)
dev.off()
\end{verbatim}

\end{document}

%%% Local Variables:
%%% mode: latex
%%% TeX-master: t
%%% End:

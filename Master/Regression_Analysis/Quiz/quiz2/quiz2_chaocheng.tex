\documentclass[a4paper,12pt]{article}
\usepackage{geometry}
\geometry{left=2.5cm,right=2.5cm,top=2.5cm,bottom=2.5cm}
\renewcommand{\textfraction}{0.15}
\renewcommand{\topfraction}{0.85}
\renewcommand{\bottomfraction}{0.65}
\renewcommand{\floatpagefraction}{0.60}
\usepackage{amsmath}
\usepackage{amsfonts}
\usepackage{mathrsfs}
\usepackage{amsthm}
\usepackage{extarrows}
\usepackage{bm}
\usepackage{graphicx}
\usepackage[section]{placeins}
\usepackage{flafter}
\usepackage{array}
\usepackage{caption}
\usepackage{subcaption}


\DeclareMathOperator*{\argmaxdown}{arg\,max}
\DeclareMathOperator*{\argmindown}{arg\,min}
\DeclareMathOperator{\argmax}{arg\,max}
\DeclareMathOperator{\argmin}{arg\,min}


\title{Quiz2 for Regression Analysis}
\author{Chao Cheng\quad No.2014210744}
\date{\today}



\begin{document}
\maketitle

\section{Weighted least squares}
The data (\texttt{wls.txt}) have two variables $x$ and $y$. There is a weight vector \texttt{w}

\begin{enumerate}
\item Draw the scatter plot of the data.
\par
See Figure~\ref{fig:w.scatter} for the scatter plot of the data.

\begin{figure}[htbp]
  \centering
  \includegraphics[width=0.6\textwidth]{W_scatter}
  \caption{Scatter plot and the weighted least square line}
  \label{fig:w.scatter}
\end{figure}


\item Write down a linear regression model for the data. Be careful because there is a weighted least squares example.
\par
The weighted linear regression model could be writtten in matrix form
\[
  Y=X\beta+\epsilon,\quad \epsilon\sim\sigma^2 W
\]
where $W=\mathrm{diag}\left(w_1,w_2,\cdots,w_n\right)$.

\item Fit the model and present a summary for the coefficients.
\par 
The fitted model is
\[
  \hat{Y}=0.9804+0.4813X
\]


\begin{table}[htbp]
  \centering
  \begin{tabular}{rrrrr}
  \hline
 & Estimate & Std. Error & t value & Pr($>$$|$t$|$) \\ 
  \hline
(Intercept) & 0.9804 & 0.4548 & 2.16 & 0.0449 \\ 
  x & 0.4813 & 0.1014 & 4.75 & 0.0002 \\ 
   \hline
\end{tabular}
  \caption{Summary table of the weighted least square regression model}
  \label{tab:w.summary}
\end{table}

and the summary table for this regression is presented in Table~\ref{tab:w.summary}. As we can see, the regression is significant.


\item Present the anova table.
\par
Table~\ref{tab:w.anova} is the ANOVA table.
\begin{table}[htbp]
  \centering
  \begin{tabular}{lrrrrr}
  \hline
 & Df & Sum Sq & Mean Sq & F value & Pr($>$F) \\ 
  \hline
x & 1 & 17.55 & 17.55 & 22.54 & 0.0002 \\ 
  Residuals & 18 & 14.02 & 0.78 &  &  \\ 
   \hline
\end{tabular}
  \caption{ANOVA table}
  \label{tab:w.anova}
\end{table}


\item Draw the regression line and 99\% point-wise confidence bands with $X$ in its range.
\par
See Figure~\ref{fig:w.ci} for the details.

\begin{figure}[htbp]
  \centering
  \includegraphics[width=0.6\textwidth]{W_CI}
  \caption{99\% Confidence band of the model}
  \label{fig:w.ci}
\end{figure}
\end{enumerate}


\section{Polynomial regression}

The data (\texttt{poly.txt}) have two variables: $x$, $y$.


\begin{enumerate}
\item Write down a linear regression model for the data.
\par
\[
  Y=X\beta+\epsilon, \quad \epsilon\overset{iid}{\sim} \mathrm{N}\left(0,\sigma^2\right)
\]

\item Fit the simple linear model and present a summary for the coefficients.
\par
The fitted model is
\[
  \hat{Y}=6.665+1.063X
\]
The summary table is presented in Table~\ref{tab:p.summary1}

\begin{table}[htbp]
  \centering
  \begin{tabular}{rrrrr}
  \hline
 & Estimate & Std. Error & t value & Pr($>$$|$t$|$) \\ 
  \hline
(Intercept) & 6.6655 & 1.4838 & 4.49 & 0.0015 \\ 
 $x$ & 1.0627 & 0.4692 & 2.26 & 0.0498 \\ 
   \hline
\end{tabular}
  \caption{Summary table for the simple linear model}
  \label{tab:p.summary1}
\end{table}


\item Fit the quadratic model and present a summary for the coefficients.
\par
The fitted model is
\[
  \hat{Y}=1.7482+1.0627X+0.4917X^2
\]
and the summary table is presented in Table~\ref{tab:p.summary2}.
\begin{table}[htbp]
  \centering
  \begin{tabular}{rrrrr}
  \hline
 & Estimate & Std. Error & t value & Pr($>$$|$t$|$) \\ 
  \hline
(Intercept) & 1.7482 & 0.5222 & 3.35 & 0.0101 \\ 
  $x$ & 1.0627 & 0.1093 & 9.72 & 0.0000 \\ 
  $x^2$ & 0.4917 & 0.0391 & 12.56 & 0.0000 \\ 
   \hline
\end{tabular}
  \caption{Summary table for the quadratic model}
  \label{tab:p.summary2}
\end{table}
\item Fit the cubic model and present a summary for the coefficients.
\par
The fitted model is
\[
  \hat{Y}=1.748205+0.999452X+0.491725X^2+0.003555X^3
\]
and the summary table is presented in Table~\ref{tab:p.summary3}

\begin{table}[htbp]
  \centering
  \begin{tabular}{rrrrr}
  \hline
 & Estimate & Std. Error & t value & Pr($>$$|$t$|$) \\ 
  \hline
(Intercept) & 1.7482 & 0.5562 & 3.14 & 0.0163 \\ 
  $x$ & 0.9995 & 0.3001 & 3.33 & 0.0126 \\ 
  $x^2$ & 0.4917 & 0.0417 & 11.80 & 0.0000 \\ 
  $x^3$ & 0.0036 & 0.0155 & 0.23 & 0.8256 \\ 
   \hline
\end{tabular}
  \caption{Summary table for cubic model}
  \label{tab:p.summary3}
\end{table}


\item Present the anova tables for three fitted models.
  \begin{table}[htbp]
    \centering
    \begin{tabular}{lrrrrr}
  \hline
 & Df & Sum Sq & Mean Sq & F value & Pr($>$F) \\ 
  \hline
x & 1 & 124.23 & 124.23 & 5.13 & 0.0498 \\ 
  Residuals & 9 & 217.97 & 24.22 &  &  \\ 
   \hline
\end{tabular}
    \caption{ANOVA table for simple linear model}
    \label{tab:p.anova1}
  \end{table}

  \begin{table}[htbp]
    \centering
    \begin{tabular}{lrrrrr}
  \hline
 & Df & Sum Sq & Mean Sq & F value & Pr($>$F) \\ 
  \hline
$x$ & 1 & 124.23 & 124.23 & 94.51 & 0.0000 \\ 
 $x^2$ & 1 & 207.46 & 207.46 & 157.82 & 0.0000 \\ 
  Residuals & 8 & 10.52 & 1.31 &  &  \\ 
   \hline
\end{tabular}
    \caption{ANOVA table for quadratic model}
    \label{tab:p.anova2}
  \end{table}

  \begin{table}[htbp]
    \centering
    \begin{tabular}{lrrrrr}
  \hline
 & Df & Sum Sq & Mean Sq & F value & Pr($>$F) \\ 
  \hline
$x$ & 1 & 124.23 & 124.23 & 83.32 & 0.0000 \\ 
 $x^2$ & 1 & 207.46 & 207.46 & 139.13 & 0.0000 \\ 
$x^3$ & 1 & 0.08 & 0.08 & 0.05 & 0.8256 \\ 
  Residuals & 7 & 10.44 & 1.49 &  &  \\ 
   \hline
\end{tabular}
    \caption{ANOVA table for cubic model}
    \label{tab:p.anova3}
  \end{table}


Please refer to Table~\ref{tab:p.anova1}, Table~\ref{tab:p.anova2} and Table~\ref{tab:p.anova3} for the ANOVA tables of these 3 models.


\item Present the scatter plot and draw three regression curves.
See Figure~\ref{fig:p.ci} for the scatter plot and 3 regression curves. Note that the cubic line are very close to the quadratic line.
\begin{figure}[htbp]
  \centering
  \includegraphics[width=0.6\textwidth]{P_CI}
  \caption{Scatter plot and 3 fitted lines of the data}
  \label{fig:p.ci}
\end{figure}


\item Choose the model that seems appropriate and justify your choice.
\par
I choose quadratic model and below are my reasons.
\par

\begin{table}[htbp]
  \centering
  \begin{tabular}{lrrrrrr}
  \hline
 & Res.Df & RSS & Df & Sum of Sq & F & Pr($>$F) \\ 
  \hline
Simple linear & 9 & 217.97 &  &  &  &  \\ 
  Quadratic & 8 & 10.52 & 1 & 207.46 & 157.82 & 0.0000 \\ 
   \hline
\end{tabular}
  \caption{ANOVA comapre quadratic and simple linear models}
  \label{tab:p.anova12}
\end{table}

Table~\ref{tab:p.anova12} compares the difference between quadratic and the simple linear models. As we can see, p-value is very small and the difference is significant.



\begin{table}[htbp]
  \centering
  \begin{tabular}{lrrrrrr}
  \hline
 & Res.Df & RSS & Df & Sum of Sq & F & Pr($>$F) \\ 
  \hline
Quadratic & 8 & 10.52 &  &  &  &  \\ 
 Cubic & 7 & 10.44 & 1 & 0.08 & 0.05 & 0.8256 \\ 
   \hline
\end{tabular}
  \caption{ANOVA compare cubic and quadradtic models}
  \label{tab:q.anova23}
\end{table}

Table~\ref{tab:q.anova23} compare the difference between cubic and quadratic models using ANOVA. As we can see, p-value is 0.8256 hence the difference between these 2 models are not significant.

\par
Also, from Table~\ref{tab:p.summary2}, the regression parameters are significant. And again, Figure~\ref{fig:p.ci} supports my decision that quadratic model is adequate.


\end{enumerate}



\clearpage
\appendix

\section{R codes}
\begin{verbatim}
require(xtable)

### weighted
rm(list=ls())
Data<-read.table("wls.txt",header=T)
str(Data)
with(Data,plot(x,y))


Data.lm<-lm(y~x,data=Data,weights=w)
abline(Data.lm)

pdf("W_scatter.pdf")
with(Data,plot(x,y))
abline(Data.lm)
dev.off()


summary(Data.lm)
anova(Data.lm)

xtable(summary(Data.lm))
xtable(anova(Data.lm))

new.x<-seq(min(Data$x),max(Data$x),length.out=100)
plot.y<-predict(Data.lm,newdata=data.frame(x=new.x),interval="confidence",level=0.99)

pdf("W_CI.pdf")
with(Data,plot(x,y))
matlines(new.x,plot.y,col=c(1,2,2),lty=c(1,2,2),lwd=2)
legend(1.5,7,c("fitted line","99% confidence band"),lty=1:2,col=1:2,lwd=2)
dev.off()

###### plolynomial ######
rm(list=ls())
Data<-read.table("poly.txt",header=T)
str(Data)
Data.lm<-lm(y~x,data=Data)
summary(Data.lm)
xtable(summary(Data.lm))

## quadratic
Data.lm2<-lm(y~x+I(x^2),data=Data)
summary(Data.lm2)
xtable(summary(Data.lm2))

## cubic
Data.lm3<-lm(y~x+I(x^2)+I(x^3),data=Data)
summary(Data.lm3)
xtable(summary(Data.lm3))

### anova
anova(Data.lm)
anova(Data.lm2)
anova(Data.lm3)

xtable(anova(Data.lm))
xtable(anova(Data.lm2))
xtable(anova(Data.lm3))

### scatter plot
pdf("P_CI.pdf")
with(Data,plot(x,y))
abline(Data.lm,col=2,lwd=2)
new.x<-seq(min(Data$x),max(Data$x),length.out=100)
y1<-predict(Data.lm2,newdata=data.frame(x=new.x))
y2<-predict(Data.lm3,newdata=data.frame(x=new.x))
matlines(new.x,cbind(y1,y2),lty=1,col=3:4,lwd=2)
legend(-3,17,c("linear","quadratic","cubic"),lty=1,col=2:4,lwd=2)
dev.off()
### model choose###
anova(Data.lm,Data.lm2)
anova(Data.lm2,Data.lm3)

xtable(anova(Data.lm,Data.lm2))
xtable(anova(Data.lm2,Data.lm3))


\end{verbatim}



\end{document}


%%%%%%%%%%%%%%%%%%%%%Subfigure
\begin{figure}[htb]
\centering
\begin{subfigure}[h]{0.49\textwidth}
\centering
\includegraphics[width=\textwidth]{5_23contour}
\caption{Contour Graphic}
\end{subfigure}
\begin{subfigure}[h]{0.49\textwidth}
\centering
\includegraphics[width=\textwidth]{5_23lattice}
\caption{Lattice Graphic}
\end{subfigure}
\caption{\label{fig:5.23.contour}}
\end{figure}

%%%%%%%%%%%%%%%%%%%%%TukeyHSD
\par
\makeatletter\def\@captype{figure}\makeatother
\begin{minipage}{.45\textwidth}
\centering
\includegraphics[width=\textwidth]{5_10tukeyplot}
\caption{Comparing Yield according to Pressure}
\label{fig:5.10.tukeyplot}
\end{minipage}
\makeatletter\def\@captype{table}\makeatother
\begin{minipage}{.45\textwidth}
\centering
\begin{tabular}{rrrrr}
  \hline
 & diff & lwr & upr & p adj \\
  \hline
215-200 & 0.32 & 0.11 & 0.52 & 0.00 \\
  230-200 & -0.18 & -0.39 & 0.02 & 0.08 \\
  230-215 & -0.50 & -0.70 & -0.30 & 0.00 \\
   \hline
\end{tabular}
\caption{Tukey multiple comparisons}
\label{tab:5.10.tukeytable}
\end{minipage}



%%%%%%%%%%%%%%%%%%%%%%%%custimize \item
\newcounter{Lcount}
\setcounter{Lcount}{0}
\begin{list}{2.1.\arabic{Lcount}}{\usecounter{Lcount}}
\item
\arabic 1, 2, 3 ...
\alph a, b, c ...
\Alph A, B, C ...
\roman i, ii, iii ...
\Roman I, II, III ...
\fnsymbol �Ǻţ������ţ�˫���ŵ�


\end{list}

%%% Local Variables:
%%% mode: latex
%%% TeX-master: t
%%% End:

\documentclass[12pt]{article}
\usepackage{amsmath}
\usepackage{amssymb}
\usepackage{amsthm}
\usepackage{mathrsfs,bm}
\usepackage{natbib}
\usepackage{graphicx}
%\usepackage[margin=1in]{geometry}
\usepackage{setspace}
\doublespacing

\addtolength{\belowcaptionskip}{4pt}
\DeclareMathOperator*{\argmax}{argmax}

\title{Quiz 3}
\author{Name:}% Put your name here

\begin{document}

\maketitle

\section{Indicator variables}
The data (\texttt{dat1.txt}) have two variables $x$ and $y$. There is a vector \texttt{set},
indicating  that there are two sets of variables.

\begin{enumerate}
\item Read the data into an R session.
\item Draw the scatter plot of the data.
\item Write down a linear regression model for the data.
\item Fit the model and present a summary for the coefficients.
\item Present the anova table.
\item Draw the regression line and 99\% point-wise confidence bands with $x$ in its range.
\item One says that, "I do not know that whether I should fit two straight lines or one straight line".
Please answer his question by re-analyzing the data.
\end{enumerate}


\section{Response transformation}

The data (\texttt{dat2.txt}) have two variables: $x$, $y$.


\begin{enumerate}
\item Read the data into an R session.
\item Write down a linear regression model for the data.
\item Fit the simple linear model and present a summary for the coefficients.
\item Perform the Box-Cox transformation
\item Find the value of $\lambda$ that maximizes the profile log-likelihood 
\item Present the profile log-likelihood plot
\item Determine the appropriate value of $\lambda$ from the profile log-likelihood plot
and justify your choice. 
\item Analyze the model with the chosen transformation.
\end{enumerate}


\appendix
\begin{verbatim}
%Put your R codes here!
\end{verbatim}
\end{document}

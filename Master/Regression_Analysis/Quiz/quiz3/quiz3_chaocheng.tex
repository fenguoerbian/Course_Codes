\documentclass[a4paper,12pt]{article}
\usepackage{geometry}
\geometry{left=2.5cm,right=2.5cm,top=2.5cm,bottom=2.5cm}
\renewcommand{\textfraction}{0.15}
\renewcommand{\topfraction}{0.85}
\renewcommand{\bottomfraction}{0.65}
\renewcommand{\floatpagefraction}{0.60}
\usepackage{amsmath}
\usepackage{amsfonts}
\usepackage{mathrsfs}
\usepackage{amsthm}
\usepackage{extarrows}
\usepackage{bm}
\usepackage{graphicx}
\usepackage[section]{placeins}
\usepackage{flafter}
\usepackage{array}
\usepackage{caption}
\usepackage{subcaption}


\DeclareMathOperator*{\argmaxdown}{arg\,max}
\DeclareMathOperator*{\argmindown}{arg\,min}
\DeclareMathOperator{\argmax}{arg\,max}
\DeclareMathOperator{\argmin}{arg\,min}


\title{Quiz3 for Regression Analysis}
\author{Chao Cheng\quad No.2014210744}
\date{\today}



\begin{document}
\maketitle

\section{Indicator variables}
The data (\texttt{dat1.txt}) have two variables $x$ and $y$. There is a vector \texttt{set},
indicating  that there are two sets of variables.

\begin{enumerate}
\item Draw the scatter plot of the data.
\par
\begin{figure}[htbp]
  \centering
  \includegraphics[width=0.7\textwidth]{1_scatter}
  \caption{Scatter plot of the data}
  \label{fig:1_scatter}
\end{figure}
See Figure~\ref{fig:1_scatter} for the scatter plot of the data.


\item Write down a linear regression model for the data.
\par
As we can see, all the data seems to lie in one straight line. So I write the linear model
\[
  y_i=\beta_0+\beta_1 x_i + \epsilon_i,\quad \epsilon_i\overset{i.i.d}{\sim} \mathrm{N}\left(0,\sigma^2\right)
\]


\item Fit the model and present a summary for the coefficients.
\par
The fitted model is
\[
  \hat{y}=1.0750+0.4923x
\]
and the summary table of the data is presented in Table~\ref{tab:1.summary}. As we can see, all the estimates are significant.
\begin{table}[htbp]
  \centering
  \begin{tabular}{rrrrr}
  \hline
 & Estimate & Std. Error & t value & Pr($>$$|$t$|$) \\ 
  \hline
(Intercept) & 1.0750 & 0.2123 & 5.06 & 0.0023 \\ 
  x & 0.4923 & 0.0286 & 17.24 & 0.0000 \\ 
   \hline
\end{tabular}
  \caption{Summary table of the linear model}
  \label{tab:1.summary}
\end{table}

\item Present the anova table.
\par
See Table~\ref{tab:1.anova} for the ANOVA of the model.

\begin{table}[htbp]
  \centering
  \begin{tabular}{lrrrrr}
  \hline
 & Df & Sum Sq & Mean Sq & F value & Pr($>$F) \\ 
  \hline
x & 1 & 25.21 & 25.21 & 297.22 & 0.0000 \\ 
  Residuals & 6 & 0.51 & 0.08 &  &  \\ 
   \hline
\end{tabular}
  \caption{ANOVA of the model}
  \label{tab:1.anova}
\end{table}

\item Draw the regression line and 99\% point-wise confidence bands with $x$ in its range.
\par
See Figure~\ref{fig:1.CI} for the 99\% confidence band plot of the data.

\begin{figure}[htbp]
  \centering
  \includegraphics[width=0.7\textwidth]{1_confidence}
  \caption{99\% confidence interval of the regression line}
  \label{fig:1.CI}
\end{figure}

\item One says that, "I do not know that whether I should fit two straight lines or one straight line".
Please answer his question by re-analyzing the data.
\par
We can fit the full model
\[
  Y_i=\beta_0+\beta_1X_i+\beta_2Set_i+\beta_3X_i:Set_i+\epsilon_i,\quad \epsilon_i\overset{i.i.d}{\sim} \mathrm{N}\left(0,\sigma^2\right)
\]
and compare this full model with our previously restricted model using ANOVA. The result is presented in Table~\ref{tab:1.anovacompare}. As we can see, the p-value is 0.4134, a faily large value. Hence we can not reject the null hypothesis, which means the restricted model is as good as the full model. 
\par
At this point, we can conlude that we do not need to fit 2 straight lines.


\begin{table}[htbp]
  \centering
  \begin{tabular}{lrrrrrr}
  \hline
 & Res.Df & RSS & Df & Sum of Sq & F & Pr($>$F) \\ 
  \hline
Restricted model & 6 & 0.51 &  &  &  &  \\ 
  Full model & 4 & 0.33 & 2 & 0.18 & 1.11 & 0.4134 \\ 
   \hline
\end{tabular}
  \caption{ANOVA comparing two models}
  \label{tab:1.anovacompare}
\end{table}
\end{enumerate}


\section{Response transformation}
The data (\texttt{dat2.txt}) have two variables: $x$, $y$.


\begin{enumerate}
\item Write down a linear regression model for the data.
  \begin{figure}[htbp]
    \centering
    \includegraphics[width=0.7\textwidth]{2_scatter}
    \caption{Scatter plot of the data}
    \label{fig:2.scatter}
  \end{figure}
\par
As we can see from Figure~\ref{fig:2.scatter}, there is clearly an up trend in the data. So I choose to fit a polynomial model.
\[
  y_i=\beta_0+\beta_1x_i+\beta_2x_i^2+\epsilon_i,\quad \epsilon_i\overset{i.i.d}{\sim} \mathrm{N}\left(0,\sigma^2\right)
\]

\item Fit the simple linear model and present a summary for the coefficients.
\par
The fitted model is
\[
  \hat{y}=848.23-440.32x+57.57x^2
\]
and the summary table is presented in Table~\ref{tab:2.summary}. The quadratic term in this model is significant, hence the whole model is significant.


\begin{table}[htbp]
  \centering
  \begin{tabular}{rrrrr}
  \hline
 & Estimate & Std. Error & t value & Pr($>$$|$t$|$) \\ 
  \hline
(Intercept) & 848.2293 & 569.5061 & 1.49 & 0.1442 \\ 
  x & -440.3181 & 247.3884 & -1.78 & 0.0827 \\ 
  I(x\verb|^|2) & 57.5659 & 26.7817 & 2.15 & 0.0377 \\ 
   \hline
\end{tabular}
  \caption{Summary of the polynomial model}
  \label{tab:2.summary}
\end{table}



\item Perform the Box-Cox transformation
  \begin{figure}[htbp]
    \centering
    \includegraphics[width=0.7\textwidth]{2_boxcox}
    \caption{Box-Cox transformation of the response}
    \label{fig:2.boxcox}
  \end{figure}
\par
Figure~\ref{fig:2.boxcox} is the Box-Cox transformation of the response.


\item Find the value of $\lambda$ that maximizes the profile log-likelihood 
\par
The value of $\lambda$ that maximizes the profile log-likelihood is 
\[
  \hat{\lambda}=0.4747
\]
\item Present the profile log-likelihood plot
\par
See Figure~\ref{fig:2.boxcox} for the details.

\item Determine the appropriate value of $\lambda$ from the profile log-likelihood plot
and justify your choice. 
\par
I think $\lambda_0=0.5$ is a good choice since it is within the 95\% confidence interval and very convenient to compute.

\item Analyze the model with the chosen transformation.
\par
The transformed response is
\[
  v_i=\dot{y}^{1-\lambda_0}\left(y_i^{\lambda_0}-1\right)/\lambda_0
\]
where $\dot{y}=\left(\prod\limits_{i=1}^ny_i\right)^{1/n}$
\par
The fitted model is
\[
  \hat{v}=-348.86+91.52x
\]
The summary table is presented in Table~\ref{tab:2.transform}. As we can see, the regression is significant. Also refer to Figure~\ref{fig:2.scatter2}. The regression line fitted the data very well. What's more, some diagnostic figures are shown in Figure~\ref{fig:2.diag}. As far as we concern, the fitted seems to be good, although I have to say the first 2 Cook's distance is a little too large for the data set.

\begin{table}[htbp]
  \centering
  \begin{tabular}{rrrrr}
  \hline
 & Estimate & Std. Error & t value & Pr($>$$|$t$|$) \\ 
  \hline
(Intercept) & -348.8626 & 38.6390 & -9.03 & 0.0000 \\ 
  x & 91.5211 & 8.1577 & 11.22 & 0.0000 \\ 
   \hline
\end{tabular}
  \caption{Summary table of the transformed response model}
  \label{tab:2.transform}
\end{table}

\begin{figure}[htbp]
  \centering
  \includegraphics[width=0.7\textwidth]{2_fitted}
  \caption{Scatter plot of the transformed data}
  \label{fig:2.scatter2}
\end{figure}


\begin{figure}[htbp]
  \centering
  \includegraphics[width=\textwidth]{2_diagnostic}
  \caption{Some diagnostic of the transformed model}
  \label{fig:2.diag}
\end{figure}
\end{enumerate}








\clearpage
\appendix

\section{R codes}
\begin{verbatim}
require(xtable)
require(MASS)

rm(list=ls())
Data<-read.table("dat1.txt",header=T)
str(Data)

### scatter plot ###
pdf("1_scatter.pdf")
with(Data,plot(x,y,type="n"))
with(Data[Data$set==1,],points(x,y,pch=2,col=2))
with(Data[Data$set==2,],points(x,y,pch=3,col=3))
legend(0,6,c("set=1","set=2"),pch=2:3,col=2:3)
dev.off()
### linear model ###
Data.lm<-lm(y~x,data=Data)
xtable(summary(Data.lm))
xtable(anova(Data.lm))

### confidence band ###
new.x<-seq(min(Data$x),max(Data$x),length.out=10)
plot.y<-predict(Data.lm,data.frame(x=new.x),interval="confidence",level=0.99)
matlines(new.x,plot.y,col=c(1,4,4),lty=c(1,2,2))

### dummy variable ###
Data.lm2<-lm(y~x+set+set:x,data=Data)
xtable(anova(Data.lm,Data.lm2))



###### 2 ######
rm(list=ls())
Data<-read.table("dat2.txt",header=T)
str(Data)

### linear ###
with(Data,plot(x,y))
Data.lm<-lm(y~x+I(x^2),data=Data)
xtable(summary(Data.lm))

new.x<-seq(min(Data$x),max(Data$x),length.out=20)
plot.y<-predict(Data.lm,data.frame(x=new.x))
lines(new.x,plot.y)


plot(Data.lm)

### boxcox ###
lambda<-boxcox(Data.lm,lambda=seq(0,1,0.05))
lambda$x[which.max(lambda$y)]

### appropiate lambda ###
l0<-0.5
n<-length(Data$y)
ydot<-prod(Data$y)^(1/n)
v<-ydot^(1-l0)*(Data$y^l0-1)/l0
Data.lm2<-lm(v~x,data=Data)
xtable(summary(Data.lm2))
anova(Data.lm2)

plot(Data$x,v,xlab="x",ylab="transformed response")
abline(Data.lm2)

plot(Data.lm2)

res<-resid(Data.lm2)
res_t<-rstudent(Data.lm2)
res_s<-rstandard(Data.lm2)
res_c<-cooks.distance(Data.lm2)

plot(abs(res),type="h")
plot(abs(res_t),type="h")
plot(abs(res_s),type="h")
plot(res_c,type="h")


\end{verbatim}


\end{document}


%%%%%%%%%%%%%%%%%%%%%Subfigure
\begin{figure}[htb]
\centering
\begin{subfigure}[h]{0.49\textwidth}
\centering
\includegraphics[width=\textwidth]{5_23contour}
\caption{Contour Graphic}
\end{subfigure}
\begin{subfigure}[h]{0.49\textwidth}
\centering
\includegraphics[width=\textwidth]{5_23lattice}
\caption{Lattice Graphic}
\end{subfigure}
\caption{\label{fig:5.23.contour}}
\end{figure}

%%%%%%%%%%%%%%%%%%%%%TukeyHSD
\par
\makeatletter\def\@captype{figure}\makeatother
\begin{minipage}{.45\textwidth}
\centering
\includegraphics[width=\textwidth]{5_10tukeyplot}
\caption{Comparing Yield according to Pressure}
\label{fig:5.10.tukeyplot}
\end{minipage}
\makeatletter\def\@captype{table}\makeatother
\begin{minipage}{.45\textwidth}
\centering
\begin{tabular}{rrrrr}
  \hline
 & diff & lwr & upr & p adj \\
  \hline
215-200 & 0.32 & 0.11 & 0.52 & 0.00 \\
  230-200 & -0.18 & -0.39 & 0.02 & 0.08 \\
  230-215 & -0.50 & -0.70 & -0.30 & 0.00 \\
   \hline
\end{tabular}
\caption{Tukey multiple comparisons}
\label{tab:5.10.tukeytable}
\end{minipage}



%%%%%%%%%%%%%%%%%%%%%%%%custimize \item
\newcounter{Lcount}
\setcounter{Lcount}{0}
\begin{list}{2.1.\arabic{Lcount}}{\usecounter{Lcount}}
\item
\arabic 1, 2, 3 ...
\alph a, b, c ...
\Alph A, B, C ...
\roman i, ii, iii ...
\Roman I, II, III ...
\fnsymbol �Ǻţ������ţ�˫���ŵ�


\end{list}

%%% Local Variables:
%%% mode: latex
%%% TeX-master: t
%%% End:
